\section{Conclusion}
\label{ch4-sec:conclusion}


In this chapter, we proposed a new problem called task-specific region partition. The problem is motivated by the fact that existing administrative boundaries are static regardless of the target variable, and we observed cases where it is necessary to have different partitions for different tasks. The task-specific region partition problem is NP-hard, and hence directly searching for a global optimal is difficult. Three variants of MCMC methods are proposed to solve this combinatorial optimization problem. First, a Naive MCMC that generates the next sample by random sampling. Second, a heuristic-based, Guided MCMC method that prefers to select from community areas with larger errors to generate the next sample. Finally, we employ reinforcement learning to automatically learn a sample strategy. Our methods are evaluated on two prediction tasks, i.e. crime prediction and real estate price prediction. The learned predictions consistently outperform the administrative boundaries in both tasks.