% !TEX root =  main.tex
\section{Introduction}

% More data are available
Recent years have seen more and more cities join the open data initiative. For example, as of December 2016, more than 1600 data sets are available on  NYC open data catalog \cite{nyc-open-data}. These public urban data collectively reflect urban dynamics and provide a unique opportunity to fully engage the data-informed city. 

% model

Given the growing amount of urban data, a number of data-driven models have been developed to provide insights for urban problems. A frequent approach is to treat each region as a data sample, take the region properties as features, and build a model to learn the correlation between region features and a target variable. One common application lies in the domain of crime prediction. Criminologists are interested in knowing the correlation between demographics and crime~\cite{wang2016crime,wang2017non}. Each region $i$ is taken as a data sample, with $X_i$ as its demographic feature and $Y_i$ as its crime count. A model (e.g., linear regression or negative binomial model) is built to estimate crime count vector $Y$ using the feature matrix $X$. If some features (e.g., disadvantage index) show a significant correlation with crime count, researchers could relate this empirical result with criminology theory~\cite{graif2014urban} and policy makers could further propose corresponding policies to address crime issues.

Existing studies often use pre-defined administrative boundaries (e.g., street block, census tract, community area, or neighborhood) to define a region~\cite{yuan2012discovering,wang2016crime}. An example of the 77 administrative community areas of Chicago is shown in Figure~\ref{fig:intro}. However, such administrative boundaries might be too rigid and do not reflect the true spatial structure with regards to the targeted urban issues. For example, the definition of regions for crime study might be different from the regions used to understand real estate market. One can alternatively propose to study the problem at the point level or small grid cell level (e.g., 10 meters by 10 meters grids). However, this could lead to data sparsity issues and jeopardize the integrity of the statistical model. Consequently, any inference made from the model would be at risk of bias. These concerns have both theoretical and policy implications. We use the following example to further illustrate the issue of using pre-defined community areas to study crime.

\begin{figure}[t]
\centering
\includegraphics[width=0.4\linewidth]{fig/error_heatmap.pdf}
\includegraphics[width=0.5\linewidth]{fig/ca-abs-errors.pdf}
\caption{Crime prediction error at community level in Chicago. The community area \#6 is an outlier with a large error.}
\label{fig:intro}
\end{figure}

\begin{figure}
\centering
\subfigure[Census tracts of community \#6.]{\includegraphics[width=0.5\linewidth]{fig/tracts_within_CA.pdf}}
\subfigure[Tract similarity.]{\includegraphics[width=0.45\linewidth]{fig/tract_sim_matrix.pdf}}
\caption{Explaining the outlying community area \#6 in Figure~\ref{fig:intro}. (a) Visualization of the 34 tracts in community \#6. (b) The pair-wise similarity of 34 tracts in terms of demographic features. It is clear that five tracts (in red color) on the east side are different from the other blue tracts.}
\label{fig:intro-explain}
\end{figure}

\begin{example}
Following previous work~\cite{wang2016crime}, we construct a negative binomial model to predict crime count using demographic features by treating each community area as a data sample. Figure~\ref{fig:intro} plots the crime prediction error for each community area of Chicago. Community area \#6 (i.e., Lake View area) shows an abnormally high error. In order to explain this outlier, we further investigate the internal structure of this area. Community \#6 consists of 34 census tracts as shown in Figure~\ref{fig:intro-explain}(a).  In Figure~\ref{fig:intro-explain}(b), we visualize  pair-wise Euclidean distance between tracts based on demographic features. There are five tracts that are different from the other tracts in the demographic feature space, and they are all located on the east side of community \#6. These five tracts, when mixed with other tracts in this community, lead to inferior performance of crime count prediction in that area. 
\end{example}

% problem definition
The observation above motivates us to learn a better region definition for crime study. In this chapter, we propose a new problem of \emph{task-specific region partitioning}. Given spatial variables and a selected model  (e.g., linear regression), we aim to partition the city into regions such that the model trained by taking regions as data samples achieves the optimal results.

% literature
%two lines of literature: (1) clustering, no target; (2) partition tracts into regions, and build a model for each region.
To the best of our knowledge, task-specific region partitioning is a new problem that has not been studied before. While there exist many methods for spatial clustering~\cite{miller2009geographic}, most of them group the locations based on the similarity of their spatial properties and do not have a target variable to predict. Another similar problem is to partition the locations into $k$ regions and fit a model for each region. Such a problem definition has a different purpose with the goal of showing that the correlations between features and target vary over the space (e.g., in some area, disadvantage index  correlates with crime; while in some other areas, it does not). In our problem definition, we aim to fit one model for the whole city hoping to get a generalized interpretation (e.g., disadvantage index significantly correlates with crime count in Chicago). Such a problem definition is a frequently adopted form in the criminology literature~\cite{graif2014urban}.

% challenge & proposed solution
Task-specific city region partitioning is a challenging problem. The key challenge lies in that the region properties (both features and target variable) and the model coefficients change simultaneously when we change the region partition. We prove that this is an NP-hard problem. In our proposed solution, we employ the Markov Chain Monte Carlo (MCMC) method. We start from a pre-defined region partition (e.g., community areas), and generate a new partition sample by flipping the membership of a smaller area (e.g., a tract). Two variants of MCMC methods are proposed to solve this problem. First, a naive MCMC method generates the next partition sample by randomly flipping a tract. Second, a heuristic-based MCMC method generates the next sample by flipping one tract randomly selected from the community areas with the highest error. Finally, we employ reinforcement learning to automatically learn how to generate the next sample that is more likely to improve the prediction performance.


% Experiments results
We evaluate our method on two real datasets, i.e. crime count and real estimate price. The learned region partitions are shown to consistently outperform the administrative boundaries and spatial clustering method. For example, our methods, on average, outperform the administrative boundary by $56\%$ in a crime prediction task. We also observe that the heuristic-based MCMC converges faster than the naive MCMC, while the reinforcement learning uses the least iterations to converge.

% Key contributions
To summarize, the key contributions of this chapter are:
\begin{itemize}[leftmargin=*]
\item We propose a novel problem on task-specific region partitioning. This problem is motivated by real-world urban studies, including our own previous work on crime prediction~\cite{wang2016crime}.
\item We prove the problem is NP-hard and we study different MCMC and reinforcement learning sampling techniques to solve the problem. 
\item We validate our method through extensive experiments on two real datasets.
\end{itemize}


% chapter organizations
The rest of this chapter is organized as follows. The formal definition of our problem is given in Section~\ref{ch4-sec:problem}, and our method is described in Section~\ref{ch4-sec:method}. Section~\ref{ch4-sec:experiment} shows the evaluations on two different tasks. Section~\ref{ch4-sec:related-work} summarizes related work.  Finally, we conclude in Section~\ref{ch4-sec:conclusion}.

