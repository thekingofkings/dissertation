\section{Related Work}
\label{sec:related-work}


\textbf{Urban Data Heterogeneity.} Various urban data usually are correlated. As we collect more types of new urban data, we are able to solve a wide spectrum of urban problems. For example, a real-time air quality inference system is proposed in \cite{zheng2013u}, which uses not only historical air quality data, but also traffic flows, structure of roads, and POIs. Zheng et. al.~\cite{zheng2014diagnosing} diagnose New York City noise pollution with complain records, road networks, and human check-ins. The real estate values are predictable given online user reviews~\cite{fu2014sparse} and offline human mobility data~\cite{wang:region}. Wang et. al.~\cite{wang2016crime, wang2017non} improve crime prediction accuracy by combining POI data and taxi flow data.

These existing works focus on mining the subtle correlations across different domains of data. We generalize the urban problems above as a learning task $f$, which takes some urban features as input and predicts the other. In this paper, we use crime prediction and average house price prediction as two examples. We study how to define the domain of urban problem $f$, because only when $f$ is defined over a proper unit of study (e.g. community areas), the learned correlation is consistent and significant.\\


\noindent\textbf{Traditional Region Partition Methods.} Our problem falls into the region segmentation category. There are mainly four types of region partition methods that are widely used in the urban computing literature. First, a fixed sized grid is the most straightforward partition for travel time prediction~\cite{wang2016simple}, interpret traffic dynamics~\cite{wu2016interpreting}, and air quality inference~\cite{zheng2013u}. Second, existing administrative boundaries are also used for crime prediction~\cite{wang2016crime}. Third, clustering of point-wise urban data to get region. For example, Li et. al.~\cite{li2015traffic} study the bike-sharing system and propose to estimate the supply/demand of bikes in a station cluster. Finally, other partitions are specifically designed for special needs. For example, Yuan et. al.~\cite{yuan2012discovering} employ the major road networks to partition city into regions and learn the function of each region. Xu et. al.~\cite{xu2017understanding} partition city by cellular tower coverages to study the mobile traffic pattern in urban environment. Zheng et. al.~\cite{zheng2015forecasting} use fan-shape partition to predict the fine-grained air quality, because wind direction is an important indicator.


While various partition methods are used in existing urban problems, none of the partition methods explicitly take learning task $f$ into consideration. It is worthy mentioning that most existing partition methods are purely based on cartographic information, and do not make use of the urban data properties. In this paper, we try to partition the city with an explicit objective.\\



\noindent\textbf{Discrete Optimizations.} The objective of our problem is easy to derive, which is a discrete optimization problem. However, since the problem is NP-complete, it is challenging to efficiently find optimal solution. MCMC sampling has been used to optimize discrete structures~\cite{strens2003evolutionary}. We follow this line of work and propose to use MCMC sampling to search for the optimal partition. During the MCMC sampling process, a lot of partitions are sampled but does not achieve better prediction results. To solve this issue, we follow the learning to optimize~\cite{li2016learning}, and propose to employ the reinforcement learning framework to learn where to sample next partition that is more likely to gain.