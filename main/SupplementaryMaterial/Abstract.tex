With the advent of information age, various types of data are collected in the context of urban spaces, including taxi pickups/drop-offs, tweets from users, air quality measure, noise complaints, Point-Of-Interest (POI), and many more. It is crucial to use these data to understand the mobility-flow-incurred interactions in the city. This dissertation views the city as a spatial network of communities linked by mobility flow. Research questions that I try to answer are: 1) how to understand nodes using heterogeneous links; 2) how to automatically partition continuous space to construct discrete nodes; 3) how to model spatial non-stationary property within the spatial network.

This dissertation aims at modeling the complicated interactions of regions in the urban space. Traditionally, due to lack of flow data, interaction is defined only by spatial distance. Recently, the availability of movement data enables us to study the interactions incurred by various social flow data (e.g., taxi flows, commuting flows).  While various approaches are proposed, they still have the following drawbacks. 1) The definition of interactions from different data sources is ad-hoc. 2) Using pre-defined region boundaries may not be appropriate. 3) The non-stationary spatial correlations are not well modeled.
 
In this dissertation I propose to develop a unified framework to model the mobility-flow-incurred interactions in the urban context. We start with a preliminary study on improving Chicago community level crime prediction with POI and taxi flow. Second, we propose a heterogeneous graph embedding method to incorporate various links and define region similarity. Third, we propose a method to automatically partition the city into regions, in order to preserve the consistency of correlations among different features. Finally, we tackle the spatial non-stationary property with local models instead of a global model. 





