With the advent of information age, various types of data are collected in the context of urban spaces, including taxi pickups/drop-offs, tweets from users, air quality measure, noise complaints, POIs, and many more. It is crucial to use these data to understand the mobility-flow-incurred interactions in the city. This thesis views the city as a spatial network of communities linked by mobility flow. Research questions that this thesis tries to answer are: 1) understand nodes using links; 2) identify spatial non-stationary property; 3) automatically partition regions as nodes.

This thesis aims at modeling the complicated interactions of regions in the urban space. Traditionally, due to lack of flow data, interaction is defined only by spatial distance. Recently, the availability of movement data enables us to study the interactions incurred by various social flow data (e.g., taxi flows, commuting flows).  While various approaches are proposed, they still have the following drawbacks. 1) The definition of interactions from different data sources is ad-hoc. 2) Most models assume uniform correlation among different spatial regions.
 
This thesis proposes to develop a unified probabilistic graphical model to model the mobility-flow-incurred interactions in the urban context. We start with a preliminary study on improving the Chicago community level crime prediction with POI and taxi flow. Second, We propose a heterogeneous graph embedding method to incorporate various links and define region similarity. Third, we tackle the spatial non-stationary property with local models instead of a global model. Finally, we propose a method to automatically partition the city into regions, in order to preserve the consistency of correlations among different features.





