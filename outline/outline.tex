\documentclass{article}


\usepackage[top=1in,bottom=1in,left=1in,right=1in]{geometry}



\begin{document}


\section{Introduction}

In the urban space, explain the data variatey, volume, and velocity. They pose challenges and opportunities in addressing real urban computing problem.

Challenges:
\begin{itemize}
\item Data sparse and nosise
\item Heterogeneous data type
\item Data is huge
\end{itemize}

Problems with existing methods:
\begin{itemize}
\item model of interaction is ad-hoc
\item the spatial non-stationarity
\end{itemize}


My reserach focus is to model complicated interactions in the urban space. It could be interactions of regions, people, and features. And I want to propose a unified graphical model to do this.

\subsection{Research questions include:}

\begin{itemize}
\item How does the crime (or any other feature) in one region correlate and influence that variable of other regions.
\item What influence crime in focal region, the crime in its neighbor, or the demographics in its neighbor.
\item Measure the similarity of two person, given their observations.
\item How would the traffic change, if build a new transportation center at location X.
\end{itemize}


Understanding the interactions can help us solve three fundamental problems in urban space.

\begin{itemize}
\item Inference (single source from others) problem. [Prediction]
\item Structural learning. Does two variable correlated? How?
\item Partition regions (address misalignment issue). community detection based on high-dimension similarity.
\end{itemize}



\subsection{Why use grapihcal model?}

A graphical model support three kinds of learning problem:

\begin{enumerate}
\item Infer unobserved values.
\item Learn parameters.  Given graph, learn dependency.
\item Structural learning. Given data, the graph is too complicated to build. Learn the graph.
\end{enumerate}

Therefore,  a graphical model is a unified approach.






\section{Inference Problem}

\subsection{Take crime inference as example}

Spatial autogressive model


Enhance with 1) newer type of data, and 2) Negative binomial regression.



Issues with current model

\begin{itemize}
\item Spatial non-stationarity  $\rightarrow$ adaptive model
\item Simplified assumption  $\rightarrow$ graphical model
\end{itemize}


In the literature, GWR solves the non-stationarity problem.

What is the weakness of GWR.

\textbf{There are multiple network adjacency structure. How to learn the importance of each netowork?} Namely, spaital nearby neigbhors show higher influence to its neighbors than that of taxi flow connected neighbors.





\section{Structural Learning Problem}

The graphical model on crime network.




Model the assumptions of crime inference
\textbf{What influence the crime count in focal area? Crime in neighors or demographics in neighbors}.




\end{document}