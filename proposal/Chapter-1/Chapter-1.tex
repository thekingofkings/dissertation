%SourceDoc ../YourName-Dissertation.tex
\chapter{Introduction -- Urban Computing} \label{chapter1:introduction}





\section{Model Mobility-Flow-Incurred Interactions}


In the urban space, the movement of human population connects two disjoint regions, and brings influence from one to the other. My research focuses on modeling the complicated interactions in the urban space with human mobility data.  


In recent years, there are some large datasets of urban taxi made public~\cite{nyctaxi} under the Freedom of information request law.  The taxi dataset contains the time and location of pick-up and drop-off for a trip. By aggregating the taxi data, we are able to get the mobility flow among different regions.
Mobility flow data (e.g., commuting flow, taxi trajectories) are sensitive resources that urban planners can use to address city issues.




Consider the following two examples. These two examples show that it is not sufficient to only consider spatial interaction, and the interactions incurred by mobility flow is equally important.


\textbf{Example 1.} \emph{Policy makers are deciding where to construct a shelter for families that are victims of violence. They understand the value of locating the shelter geographically far from violent neighborhoods. One possible choice is to locate the shelter in a neighborhood that is 10 miles from the violent neighborhoods where vulnerable families previously lived. However, a deeper analysis may reveal that the new neighborhood, though geographically removed from the old neighborhood, may still have strong social flows (connections caused by commutes, family visits) with the old neighborhood. Emerging research suggests that a great deal of crime happens in areas that are socially connected to offenders' neighborhoods. This suggests that shelters may benefit from being located in a neighborhood that is also socially isolated from violence (e.g., with weak communication and commuting interactions with the violent neighborhoods that shelter residents fled from) while socially connected to jobs, services, and resources.}


\textbf{Example 2.} \emph{In order to provide better living environment to crowded city, city planner decide to expand city with a new satellite town. However, people are not willingly to move, if there is not enough incentive. To create opportunity and attract people, some candidates to relocate are big factory, shopping mall, and government offices. The question is which facility should be relocated, so that population pressure in whole  city  is relieved.  Correspondingly, how many new residential building, supermarket, and parks should be built into the new city? People have to go the new city to work. Some of them may move to live in the new city, and the rest may choose to commute everyday.}





\section{Research Problems}
\label{sec:qa}


In the urban space, a model of region \textbf{interactions} can help solve  several fundamental data mining questions. This thesis views  the city as a spatial network of communities linked by ``hyperlink'' flow. The questions we want to answer are:
\begin{itemize}
\item Understand nodes using links. Estimate an unobserved property of focal community, given the observations on other communities and other types of data. For example, the crime rate in a residential neighborhood could be impacted by  non-adjacent but flow-connected neighborhoods, because the residents in the neighborhoods are also exposed to and influenced by the environment in the workplace.
\item Understand links using nodes. Are certain properties of two connected nodes associated with the type and volume of the flow connections? For example, given the crime profiles of two connected communities, which type of interactions (taxi, LEHD, or space continuity) is more important in forming the crime properties?
\item Identify the fundamental dependency structure among properties of nodes. For example, the crime rate of two connected communities may show a strong correlation. However, this does not necessarily mean that high crime rate in one community lead to the high crime in another. The crime is very likely to be caused by other properties. 
\end{itemize}




\section{Existing Work}
\label{sec:ew}



\subsection{Spatial Interaction Model}


Spatial interaction is a broad term encompassing any movement over space that results from a human process \cite{haynes1984gravity,rodrigue2013geography}. It measures the flow between an origin and a destination, given distance and nodal properties on the origin and destination.  There are several classic models to measure the spatial interactions. For example, \textbf{gravity model} \cite{matyas1997proper} uses a similar formulation than Newton's law of gravity to model the interaction between two regions.
\textbf{Spatial autoregressive model} \cite{anselin1980estimation} is another widely used inference model that infer a property in a region with nearby region's information. There are a lot of  applications  based on the spatial interactions, such as international trade \cite{carrere2006revisiting, egger2003proper, martinez2003augmented}, population migration \cite{hanski1994metapopulation, karemera2000gravity, lewer2008gravity}, traffic flow \cite{jung2008gravity, roughan2002experience, khadaroo2008role}, telecommunication flow \cite{krings2009urban, fischer1994artificial, black1995spatial}, crime estimation \cite{anselin2000spatial, kakamu2008spatial, browning2004paradox}, knowledge spillover \cite{lesage2007knowledge, fischer2006geography}, and many more.



Notice that the region interaction in this thesis proposal is different from the traditional \textbf{spatial interaction} study, which assumes the interaction of two regions is reversely correlated with their geographical distance. However, space is a biased sample on human mobility, and should not be the only measure to model human movement. For example, during the weekdays people usually follow a regular home-office commuting pattern. The workplace region might be far from home region, but their connection is strong due to the human movement. As there are more and more mobility data available, we propose to employ the flow data in the interaction model as well.







\subsection{Exploring Mobility Flow}

In recent years, the availability of flow data enables research progress on exploring the role of various flows. New observations are made on the mobility data. For example, there are studies to explain epidemic spread with air travel \cite{huang2013global, tatem2014mapping} and population migration survey \cite{pindolia2013demographics}. Zheng et al. \cite{zheng2011urban} identify the underlying road network problem by detecting anomalous pair of flow. Yuan et al. \cite{yuan2012discovering} discover the function of regions by learning a topic model on the flow matrix and clustering the flow of region pairs into different topic.  Berlingerio et al. \cite{berlingerio2013allaboard} optimizes public transport route by looking at the region origin/destination flow. 



The problem in this proposal is different from the works above in the sense that different approaches are taken. Those works in the literature study the mobility flow by itself. Namely, the literature focuses on observing the properties of the mobility flow by various data mining technique, such as clustering \cite{berlingerio2013allaboard}, topic modeling \cite{yuan2012discovering}, outlier detection \cite{zheng2011urban} etc.  The approach in the literature uses the observed properties of flow to manually explain a phenomenon. This thesis proposal takes a different approach, that explicitly models the statistic dependency between social flow and region properties in an inference problem setting. The approach in this thesis allows us to answer those three research questions raised in Section~\ref{sec:qa}, which the literature methods cannot answer.





\subsection{Interaction Model in Social Networks}


In social network analysis, there is a line of work that focuses on infer an unobserved  nodal feature from neighboring nodes. Examples are infer user home location \cite{Pontes:2012:WKY:2370216.2370419, Li:2012:TSU:2339530.2339692}, user age \cite{6195711}, and more categorical features \cite{Mislove:2010:YYK:1718487.1718519}. Earlier works mainly focus on one type of features, and employ the network structure to make inference. To solve the application problem of user profiling, some techniques are borrowed from community detection \cite{fortunato2010community}, information diffusion \cite{guille2013information}, collaborative filtering \cite{breese1998empirical}, etc. The inferred features are assumed to be similar within a cluster. Therefore, community detection will cluster nodes with unobserved property together with nodes with observed property, and thus we can conduct inference. Another angle is to study how information is propagated through network. Methods under the information diffusion category is explanatory \cite{rodriguez2011uncovering, gomez2010inferring}.


With more data types available, there is one line of works solving the nodal feature inference problem with \textbf{composite social network} \cite{pan2011composite, madan2011pervasive,zhong2012comsoc}. Composite social network is a graph with multiple types of edges. For example in a social network, two users could be connected by following, how many re-tweets, how many messages are exchanged, etc. 


Our work is different from those work in social network literature in a way that we do not use observed edge strength as the interactions. All those paper in literature assumes that if two nodes are connected by a edge with heavy weight, then they are very likely to have strong interactions. This is not necessarily true. Take the crime inference as example. Suppose two communities are connected by heavy traffic flow, and both community are crime free. In this case, we cannot say two regions have strong crime interactions.







\section{Challenges}

There are mainly three challenges to address.

\textbf{Given one type of social flow, define interaction is non-trivial}. Given only one type of social flow, there are too many possibilities in constructing interactions.   1) The flow matrix can take various form. For example, we can chose normalize the flow matrix or not. When normalizing the flow matrix, we can chose whether normalize by in-flow or out-flow. 
2) There are also many nodal properties that could interact with their neighbors, such as various demographics features. 
3) Different kinds of function can be used to define interactions. Take the product of flow and nodal properties is the most straightforward choice. However, sometimes it also makes sense to further apply a distance exponential decay on previous product.
Therefore, the construction of interaction is totally ad-hoc. 


\textbf{Given multiple types of social flow, the interaction is more difficult to define}. It is possible we have multiple types of social flow (e.g. taxi flow, commuter transit). For one pair of regions, should we build separate interaction over different social flows, or sum over all flows to get one interaction? When we take the sum, should we weight different social flow differently, and how?


\textbf{The interaction could be spatial non-stationary}. Most models in existing work are global model, which assumes the statistic interaction does not vary over space. However, some urban data have spatial non-stationary property. Therefore, using global estimates of relationships can present misleading interpretations of local relationships.



\textbf{Identify discrete node from continuous space is challenging}. In the literature, the boundary of regions are usually pre-defined administrative regions, which has two major issues. First, this boundary definition is not consistent in terms of the property of interest. For example, it is very likely a community is formed on the boundary of two administrative regions. Also, the current partition does not evolve over time. In real life, we know there must be some dynamic change of communities. Second, partition continuous space is not trivial. There is a misalignment problem, which refers to the problem that different data are not collected in the same scale. For example, the crime record is at point level, while the demographics is at block level.

\section{A Unified Graphical Model}

We propose a unified graphical model to model the interactions among regions. Since various flow connect regions into various network. A graphical model is a natural way to represent the whole system. Additionally, this model can address the two drawbacks mentioned above.

To address the non-trivial definition of interaction. In the graphical model we model the interaction of two communities is a hidden variable. This hidden variable is connected with the observations on both flow and nodal properties. We can learn the interaction from the network.

To address the spatial non-stationarity. Each region is built as separate node in the graphical model. This way, the conditional probability 
\[ P(y_i | X_i) \]
is different at different regions, which is equivalent to have multiple local models to model the different relations between $y$ and $X$.





In next chapter we study the first problem - ``using links to understand nodes''. We use crime inference as an example, in which we use an enhanced spatial autoregressive model to predict crime count of a community using its neighbors. In Chapter \ref{ch:crf}. We first discuss the missing property of a spatial autoregressive model, which is the spatial non-stationarity. To address this, we propose a conditional random field based  graphical model. This model is superior than other method in the literature, such as geographically weighted regression. Finally, there is a research plan in Chapter \ref{ch:plan}.



