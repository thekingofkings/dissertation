% Place abstract below.
With the advent of information age, various types of data are collected in the context of urban spaces , including taxi pickups/drop-offs, tweets from users, air quality measure, noise complaints, POIs, and many more. It is crucial to use these data to solve urban issues, such as traffic congestion, crime prediction, and pollution. Two immediate questions are ``Are those different data correlated?'', ``How do we employ their correlation to infer one from the other?''

Urban computing has gained increasing popularity as an active research topic. These studies include profile city functions, detect traffic anomalies, predict air qualities, and location recommendation. The common challenges we face include 1) dealing with sparse and noisy data sources, and 2) handle implicit and complicated correlations. While various approaches are proposed, they still have the following drawbacks. 1) The partition of spatial-temporal space is over-simplified. Partition space into regions with road network or administrative division is widely used. However, these partitions does not always align with the data distribution. 2) Most models assume uniform correlation among different spatial regions. As a matter of fact, in my preliminary study I have observed that when training separate models on Chicago south and north, two models are different and the estimation results will be better. 3) The feature construction from different data sources is ad-hoc.

The goal of this thesis will be to develop a unified framework to capture the correlations of heterogeneous data in the urban context. Starting from a preliminary study on estimating the Chicago community level crime with POI and taxi flow. The intuition is that the POI complements the demographics features, and the taxi flow acts as a hyperlink to connect non-adjacent community areas. The results suggest that both newer type of features correlates with the crime and improves the estimation significantly. Next, I am trying to model spatial variations. Namely, the same features in different regions correlate differently. Meanwhile, a smart partition based on observed urban data is also a key component in this framework. Finally, the urban data are classified as nodal feature and dyadic feature, which belong to a spatial unit and a pair of units respectively. In my future work, I plan to build a consistent approach to model different features.
