% Place abstract below.
With the advent of information age, various types of data are collected in the context of urban spaces, including taxi pickups/drop-offs, tweets from users, air quality measure, noise complaints, POIs, and many more. It is crucial to use these data to understand the mobility-flow-incurred interactions in the city. This thesis views the city as a spatial network of communities linked by mobility flow. Research questions that this thesis tries to answer are: 1) understand nodes using links; 2) understand links using nodes; 3) identify causal structure.

This thesis aims at modeling the complicated interactions of regions in the urban space. Traditionally, due to lack of flow data, interaction is defined only by spatial distance. Recently, the availability of movement data enables us to study the interactions incurred by social flow data (e.g., taxi flows, commuting flows).  While various approaches are proposed, they still have the following drawbacks. 1) The definition of interactions from different data sources is ad-hoc. 2) Most models assume uniform correlation among different spatial regions.
 
This thesis proposes to develop a unified probabilistic graphical model to model the mobility-flow-incurred interactions in the urban context. We start with a preliminary study on estimating the Chicago community level crime with POI and taxi flow. The intuition is that the POI complements the demographics features, and the taxi flow acts as a hyperlink to connect non-adjacent community areas. The results suggest that using POI and taxi flow improves the crime estimation significantly. Next, the urban data are categorized as nodal feature and dyadic feature, which belong to a spatial region and a region pair respectively. The graphical model is a natural way to model the complicated interaction.
Lastly, model spatial variations, i.e., the same features in different regions have different parameters. In the graphical model, we use separate nodes to represent different regions, so that learned relations are locally specific.




