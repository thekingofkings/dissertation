% Place abstract below.
With the advent of information age, various types of data are collected in the context of urban spaces , including taxi pickups/drop-offs, tweets from users, air quality measure, noise complaints, POIs, and many more. It is crucial to use these data to understand region interactions in the city. Research questions that I am interested in are: 1) understand nodes using links; 2) understand links using nodes; 3) identify causal structure.

For my thesis work, I want to capture the complicated interactions of regions in the urban space. Traditionally, due to lack of flow data, spatial similarity is widely used to capture interactions. Recently, the availability of movement data enables us to study the interactions incurred by social flow.  While various approaches are proposed, they still have the following drawbacks. 1) The definition of interactions from different data sources is ad-hoc. 2) Most models assume uniform correlation among different spatial regions. As a matter of fact, in my preliminary study I have observed that when training separate models on Chicago south and north, two models are different and the estimation results will be better.

The goal of this thesis is to develop a unified probabilistic graphical model to capture the interactions of heterogeneous flow in the urban context. Starting from a preliminary study on estimating the Chicago community level crime with POI and taxi flow. The intuition is that the POI complements the demographics features, and the taxi flow acts as a hyperlink to connect non-adjacent community areas. The results suggest that both newer type of features correlates with the crime and improves the estimation significantly. Next, the urban data are categorized as nodal feature and dyadic feature, which belong to a spatial unit and a pair of units respectively. The graphical model is a natural way to capture the complicated interaction.
Lastly, model spatial variations, i.e., the same features in different regions have different parameters. In the graphical model, we use separate nodes to represent different regions, so that learned relations are locally specific.




