\documentclass[11pt]{article}

\usepackage[top=1in,left=1in,right=1in,bottom=1in]{geometry}


\title{Thesis Proposal}
\author{Hongjian Wang \qquad hxw186@ist.psu.edu}
\date{}


\begin{document}

\maketitle

Use heterogeneous urban data to solve practical problems.


\begin{table}[h]
\centering
\caption{The category of research problems.}
\begin{tabular}{|c|c|c|}
\hline
 & Individual level & Aggregated level \\ \hline
Single domain correlation & 	&  \\ \hline
Cross domain correlations &     &  \\ \hline
\end{tabular}
\end{table}


The urban data we have worked on
\begin{itemize}
\item Taxi trips
\item POI
\item Tweets (geo-tagged)
\item crime
\end{itemize}

The problem I am interested in working on.
\begin{itemize}
\item \textbf{Measure Relationship Strength} is a fundamental problem. The proposed measure can benefit the recommendation and a series of real applications. 
	\begin{itemize}
		\item \textbf{Current work} use the historical check-ins to profile user and measure the distance between a pair. 
        \item \textbf{Future work} (1) use cross-domain data (combine tweets hashtag, which reflects the individual interest). (2) generalize the measure to identify community. One interested approach is to create individual user embedding from their social network / co-location network. 
	\end{itemize}
\item \textbf{Traffic analysis} -- either predict travel time / traffic volume, or correlate traffic with event. 
	\begin{itemize}
		\item \textbf{Current work} travel time estimation. Use big data to ease the complexity of simple task.
        \item \textbf{Future work} correlate the traffic with event. Predict traffic variation with respect to external impact (weather, event, traffic accident).
	\end{itemize}
\item \textbf{Study social properties of city}
	\begin{itemize}
		\item \textbf{Current work} infer the crime count at a certain community area in Chicago. Correlate POI and taxi flow with crime propagation.
	\end{itemize}
\end{itemize}


\end{document}